\section{Analyse mathématique des systèmes \textit{SCEX\_T1} et \textit{SCEX\_T2}}
\subsection{\textit{SCEX\_T1}}
\subsubsection{Decimation cipher}
Ici, nous allons pouvoir retrouver une méthode de chiffrement dite de \enquote{\textit{décimation}}. Cette dernière se présente sous la forme suivante :
\begin{center}
    \begin{equation}
	c_t = \left\{
	    \begin{array}{lll}
		m_{t}\oplus\sigma_{t} & \mbox{si } t\equiv 0 & (\mbox{mod } 3) \\
		m_{t}\oplus\sigma_{t+1} & \mbox{si } t\equiv 1 & (\mbox{mod } 3) \\
		m_{t}\oplus\sigma_{t+2} & \mbox{si } t\equiv 2 & (\mbox{mod } 3)
	    \end{array}
	\right.
    \end{equation}
\end{center}

Le fonctionnement est très simple :

\begin{enumerate}
 \item Nous allons récupérer la position des caractères dans notre alphabet
 \item Nous allons alors multiplier/additionner/soustraire (modulo un nombre) cette position par un coefficient donné (la clé)
 \item La nouvelle position trouvée donne le chiffre du caractère clair
\end{enumerate}
Nous pouvons noter cet algorithme de la manière suivante :
\begin{center}
    \begin{equation}
	E(M)\equiv (kM+t)[n]
    \end{equation}
\end{center}
Où $k$ et $t$ sont des entiers et $k$ est premier par rapport à $n$ (le modulo utilisé, la taille de l'alphabet).\\
Pour déchiffrer un message ainsi codé, nous pouvons alors utiliser l'équation suivante :
\begin{center}
    \begin{equation}
	D(C)\equiv ((n-k)M+(n-t))[n]
    \end{equation}
\end{center}
Un des inconvénients de ce système est qu'il ne change pas la fréquence d'apparition des lettres, nous avons ici un système à permutations. Il est alors simple de reconnaître, en connaissant la langue d'origine du texte les fréquences les plus élevées. Une fois ces fréquences trouvées, nous pouvons poser deux équations à deux inconnues pour retrouver les coefficients $k$ et $t$.

\subsection{\textit{SCEX\_T2}}