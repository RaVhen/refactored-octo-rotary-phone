\section{Analyse mathématique des systèmes \textit{SCEX\_T1} et \textit{SCEX\_T2}}
\subsection{\textit{SCEX\_T1}}
\subsubsection{Decimation cipher}
Ici, nous allons pouvoir retrouver une méthode de chiffrement dite de \enquote{\textit{décimation}}. Cette dernière se présente sous la forme suivante :
\begin{center}
    \begin{equation}
	c_t = \left\{
	    \begin{array}{lll}
		m_{t}\oplus\sigma_{t} & \mbox{si } t\equiv 0 & (\mbox{mod } 3) \\
		m_{t}\oplus\sigma_{t+1} & \mbox{si } t\equiv 1 & (\mbox{mod } 3) \\
		m_{t}\oplus\sigma_{t+2} & \mbox{si } t\equiv 2 & (\mbox{mod } 3)
	    \end{array}
	\right.
    \end{equation}
\end{center}

Le fonctionnement est très simple :

\begin{enumerate}
 \item Nous allons récupérer la position des caractères dans notre alphabet
 \item Nous allons multiplier (modulo un nombre) cette position par un coefficient donné (la clé)
 \item La nouvelle position trouvée donne le chiffre du caractère clair
\end{enumerate}
Nous pouvons noter cet algorithme de la manière suivante :
\begin{center}
    \begin{equation}
	k*l\equiv c[n]
    \end{equation}
\end{center}
Où $k$ est le coefficient, $l$ le caractère clair, $c$ le caractère chiffré et $n$ le modulo utilisé.

\subsection{\textit{SCEX\_T2}}